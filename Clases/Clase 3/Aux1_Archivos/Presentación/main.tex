\documentclass{beamer}
%
% Choose how your presentation looks.
%
% For more themes, color themes and font themes, see:
% http://deic.uab.es/~iblanes/beamer_gallery/index_by_theme.html
%

% Colores sacados de http://latexcolor.com/, 
% ponerlos en usercolortheme para cambiar los colores de la presentación
\definecolor{persianred}{rgb}{0.8, 0.2, 0.2}
\definecolor{darkbyzantium}{rgb}{0.36, 0.23, 0.33}
\definecolor{burgundy}{rgb}{0.5, 0.0, 0.13}
\definecolor{deepjunglegreen}{rgb}{0.0, 0.29, 0.29}
\definecolor{darkcoral}{rgb}{0.8, 0.36, 0.27} %este ta weno
\definecolor{darkchestnut}{rgb}{0.6, 0.41, 0.38}
\definecolor{deepcarrotorange}{rgb}{0.91, 0.41, 0.17}
\definecolor{firebrick}{rgb}{0.7, 0.13, 0.13}
\definecolor{pastelred}{rgb}{1.0, 0.41, 0.38}
\definecolor{tyrianpurple}{rgb}{0.4, 0.01, 0.24}
\definecolor{orangeDeadPhysicitsSociety}{HTML}{f28165}
\definecolor{skymagenta}{rgb}{0.81, 0.44, 0.69}
\definecolor{patriarch}{rgb}{0.5, 0.0, 0.5}
\definecolor{darkgreen}{rgb}{0.09, 0.45, 0.27}

\usepackage{xcolor}
\usepackage{mathtools}
\usepackage{setspace}


\newcommand{\semitransp}[2][35]{\textcolor{fg!#1}{#2}}


\mode<presentation>
{
  \usetheme{Madrid}      % or try Darmstadt, Madrid, Warsaw, ...
  \usecolortheme[named=burgundy]{structure} %aquí se cambia el colore, 
  \usefonttheme{default}  % or try serif, structurebold, ...
  \setbeamertemplate{navigation symbols}{}
  \setbeamertemplate{caption}[numbered]
} 
\setbeamercovered{transparent=30}

\renewcommand{\thefigure}{}


\usepackage[english]{babel}
\usepackage[utf8]{inputenc}
\usepackage[T1]{fontenc}
\usepackage{svg}
\usepackage{graphicx} % figuras
\usepackage{subfigure} % subfiguras

\usepackage{geometry}
\usepackage{amsmath,amssymb}
\usepackage{setspace}
\usepackage{comment}
\usepackage{hyperref}

%la wea código
\usepackage{algorithm}
\usepackage{algpseudocode}
 
\usepackage{minted}
\setminted{frame=lines, bgcolor=red!5, fontsize=\footnotesize, breaklines}

\renewcommand{\inserttotalframenumber}{\pageref{lastslide}}
\usepackage{appendixnumberbeamer} 

\AtBeginSection[]{
  \begin{frame}
  \vfill
  \centering
  \begin{beamercolorbox}[sep=8pt,center,shadow=true,rounded=true]{title}
    \usebeamerfont{title}\insertsectionhead\par%
  \end{beamercolorbox}
  \vfill
  \end{frame}
}


\title[Auxiliar 1]{Auxiliar 1: Geometría computacional}
\subtitle{Introducción a objetos en C++}
\author[Computación GPU]{Profesora: Nancy Hitschfeld-Kahler \\ Auxiliar: Sergio Salinas}
\institute[DCC - UChile]{CC7515 - Computación en GPU}  
\date{March 27, 2023}

\usepackage{etoolbox}
\makeatletter
\patchcmd{\beamer@sectionintoc}{\vskip1.5em}{\vskip0.5em}{}{}
\makeatother


\DeclareUnicodeCharacter{2212}{-}



\begin{document}

\begin{frame}
  \titlepage
\end{frame}

\begin{frame}{ToC}
    \tableofcontents
\end{frame}

\section{Introducción a C++}

\begin{frame}{Introducción a C++}
    \begin{itemize}
        \item Desarrollado por Bjarne Stroustrup en 1983 como una extensión del lenguaje C.
        \item Lenguaje de bajo nivel con control de memoria necesarias para aprovechar la arquitectura de la GPU.
        \item Se recomienda usar el estándar de C++ moderno (C++ 11 en adelante).
    \end{itemize}
\end{frame}

\begin{frame}[fragile]{Hello world}

\begin{minted}[label=hello.cpp]{cpp}
#include <iostream>

int main() {
    std::cout << "Hello, world!" << std::endl;
    return 0;
}

//g++ hello.cpp -o hello
\end{minted}

\end{frame}

\begin{frame}[fragile]{Tipos de variables en C++}
\begin{minted}[fontsize=\small]{cpp}
int entero = 10;
float flotante = 3.14;
double doble = 3.141592653589793;
char caracter = 'A';
bool booleano = true;
long largo = 1234567890;
long long muy_largo = 1234567890123456789;
unsigned int sin_signo = 42;
unsigned long largo_sin_signo = 9876543210;
\end{minted}
\end{frame}

\begin{frame}[fragile]{Funciones en C++}
\begin{minted}[fontsize=\small]{cpp}
#include <iostream>
// Función que suma dos enteros y devuelve el resultado
int suma(int a, int b) {
    return a + b;
}
// Función que imprime el resultado de la suma
void imprimirResultado(int resultado) {
    std::cout << "El resultado de la suma es: " << resultado << std::endl;
}

int main() {
    int num1 = 5;
    int num2 = 7;
    int resultado = suma(num1, num2);
    imprimirResultado(resultado);
    return 0;
}
\end{minted}
\end{frame}


\section{Clases en C++}

\begin{frame}{Introducción a Clase en C++}
\begin{itemize}
\item En C++, un objeto es una instancia de una clase.
\item Una clase es una estructura de datos que define un conjunto de atributos y métodos que operan sobre esos atributos.
\item Los objetos son útiles para modelar entidades del mundo real y representar datos complejos.
\item Los objetos se pueden crear dinámicamente en tiempo de ejecución usando punteros y el operador \texttt{new}.
%\item La notación de punto (\texttt{.}) se utiliza para acceder a los atributos y métodos de un objeto.
\item Los objetos se eliminan usando el operador \texttt{delete} para liberar la memoria asignada a ellos.
\end{itemize}
\end{frame}

%Declaración de una clase
%

\subsection{Declaración de una clase}

\begin{frame}[fragile]{Declaración}
\begin{minted}[label=rectangulo\_ex1.cpp]{cpp}
class Rectangulo {
private:
    // atributos privados
    double ancho, alto;
public:
    // Constructor por defecto
    Rectangulo() {
        ancho = 0.0; alto = 0.0; }

    // Constructor con parámetros
    Rectangulo(double ancho, double alto) {
        this->ancho = ancho; this->alto = alto; }

    // método público para calcular el área
    double calcular_area(){
        return ancho * alto;}
};
\end{minted}
\end{frame}

\subsection{Creación de un objeto}

\begin{frame}[fragile]{Crear un objeto}
    \begin{minted}[label=rectangulo\_ex1.cpp]{cpp}
int main()
{
    //Crea un objeto en el stack
    Rectangulo fig1 = Rectangulo(1,2);
    fig1.calcular_area();

    //Crea un objecto en el heap
    Rectangulo *fig2 = new Rectangulo(4, 5);
    fig2->calcular_area();
    delete fig2;
    
    return 0;
}
    \end{minted}
\end{frame}



\begin{frame}[fragile]{Declaración y Definición de la Clase Rectangulo}
  \begin{columns}
    \begin{column}{0.5\textwidth}
      \begin{minted}[label=rectangulo.hpp]{c++}
// Declaración de la clase
class Rectangulo {
private:
  double ancho;
  double alto;

public:
  Rectangulo();
  Rectangulo(double an, double al);
  double calcular_area();
};
      \end{minted}
    \end{column}
    \begin{column}{0.5\textwidth}
      \begin{minted}[label=rectangulo.cpp]{c++}
#include "rectangulo.hpp"

// Definición de la clase
Rectangulo::Rectangulo() {
  ancho = 0;
  alto = 0;
}

Rectangulo::Rectangulo(double an, double al) {
  ancho = an;
  alto = al;
}

double Rectangulo::calcular_area() {
  return ancho * alto; }
      \end{minted}
    \end{column}
  \end{columns}
\end{frame}

\section{Constructores}

\begin{frame}[fragile]{Tipos de constructores y destructores}

\begin{minted}[label=rectangulo\_constructores.cpp]{cpp}
    // Constructor por defecto
    Rectangulo() {
        ancho = 0;  alto = 0;}

    // Constructor con parámetros
    Rectangulo(double ancho, double alto) {
        this->ancho = ancho; 
        this->alto = alto; }
        
    // Constructor copia
    Rectangulo(const Rectangulo& r) {
        this->ancho = r.ancho; 
        this->alto = r.alto; }
        
    // Destructor
    ~Rectangulo() {
        std::cout << "Se ha destruido un rectangulo." << std::endl;
    }    
\end{minted}
\end{frame}

\begin{frame}{Tipos de constructores y destructores}
    \begin{center}
    \includegraphics[height=8cm]{polylla.png}
    \end{center}
\end{frame}

\section{Overloading}

\begin{frame}[fragile]{Overloading de operadores en C++}

El \textbf{overloading de operadores} es una funcionalidad en C++ que permite a los desarrolladores definir una función que se comporta como un operador. Por ejemplo, para sumar dos números complejos, podemos sobrecargar el operador \texttt{+} de la siguiente manera:

\begin{minted}{c++}
class Complejo {
public:
    Complejo operator+(const Complejo& c) const {
        return Complejo(real + c.real, imag + c.imag);
    }
private:
    double real;
    double imag;
};
\end{minted}

En este caso, estamos sobrecargando el operador \texttt{+} para que sume dos números complejos. El operador se define dentro de la clase \texttt{Complejo} y se utiliza el operador \texttt{+} para definir la función.
\end{frame}

\begin{frame}[fragile]
\frametitle{Ejemplo de Overloading}

\begin{columns}
    \begin{column}{0.62\textwidth}
        \begin{minted}[label=Complejo.h]{c++}
class Complejo {
private:
    double real, imag;
public:
    Complejo(double r = 0, double i = 0) : real(r), imag(i) {}
    Complejo operator+(Complejo const &obj) {
        Complejo res;
        res.real = real + obj.real;
        res.imag = imag + obj.imag;
        return res;
    }
    void imprimir() {
        std::cout << real << " + " << imag << "i" << std::endl;
    }
};
        \end{minted}
    \end{column}

    \begin{column}{0.38\textwidth}
        \begin{minted}[label=main.cpp]{c++}
int main() {
    Complejo a(3.0, 4.0);
    Complejo b(2.0, 3.0);
    Complejo c = a + b;
    c.imprimir();
    return 0;
}
        \end{minted}
    \end{column}
\end{columns}

\end{frame}


\begin{frame}[fragile]{Ejemplo de friend class en C++}
\begin{columns}
  \begin{column}{0.5\textwidth}
  %\vspace{-2cm}
\begin{minted}[fontsize=\footnotesize, ]{cpp}
class Complejo {
private:
    ...
public:
    friend std::ostream& operator<<(std::ostream& out, const Complejo& c);
    friend std::istream& operator>>(std::istream& in, Complejo& c);
};

std::ostream& operator<<(std::ostream& out, const Complejo& c) {
    out << c.real << " + " << c.imag << "i";
    return out;
}
    \end{minted}
  \end{column}
  \begin{column}{0.5\textwidth}
    \begin{minted}[fontsize=\footnotesize, frame=lines, breaklines]{cpp}
std::istream& operator>>(std::istream& in, Complejo& c) {
    cin >> c.real;
    cin >> c.imag;
    return in;
}

int main() {
    Complejo c1(2, 3), c2;
    std::cout << "c1 = " << c1 << std::endl;
    std::cout << "Ingrese un número complejo: ";
    std::cin >> c2;
    std::cout << "c2 = " << c2 << std::endl;
    return 0;
}
    \end{minted}
  \end{column}
\end{columns}

\end{frame}


\section{Templates}

\begin{frame}[fragile]{Templates en C++}

Los templates son una característica de C++ que permiten escribir funciones y clases genéricas que pueden trabajar con diferentes tipos de datos sin tener que escribir múltiples versiones para cada tipo de dato.

\begin{minted}[label={maximo.hpp}]{c++}
template <typename T>
T maximo(T a, T b) {
return a > b ? a : b;
}

int a = 5, b = 10;
maximo(a, b);

double c = 3.14, d = 2.71;
maximo(c, d);
std::string s1 = "hola", s2 = "mundo";
maximo(s1, s2)
}
\end{minted}

\end{frame}

\begin{frame}[fragile]{Templates}
\begin{columns}
\column{0.4\textwidth}
\begin{minted}[label=Declaración]{c++}
template <typename T>
class Rectangulo {
private:
    T ancho;
    T alto;
public:
Rectangulo(T ancho, T alto)
{
    this->ancho = ancho;
    this->alto = alto;
}

T calcular_area() {
    return ancho * alto;
}
};
\end{minted}

\column{0.6\textwidth}
\begin{minted}[label=Llamada]{c++}
int main(){
    Rectangulo<int> rect1(4, 5);
    rect1.calcular_area();
    Rectangulo<float> rect2(2.5, 3.5);
    rect2.calcular_area();
    return 0;
}
\end{minted}
\end{columns}
\end{frame}


\section{Biblioteca estándar de C++}

\begin{frame}[fragile]{Biblioteca estándar de C++}
Conjunto de funciones, objetos y clases que proporcionan una amplia variedad de características y funcionalidades para el lenguaje C++. Por ejemplo:

\begin{itemize}
\item Entrada/salida: operaciones de entrada y salida, como leer o escribir en archivos o en la consola.
\item Contenedores: estructuras de datos para almacenar y manipular colecciones de objetos, como vectores, listas, mapas, etc.
\item Algoritmos: funciones para realizar operaciones comunes en contenedores, como ordenar, buscar, mezclar, etc.
\item Tipos de datos: tipos de datos comunes, como cadenas de caracteres, booleanos, números, etc.
\item Funciones matemáticas: funciones matemáticas comunes, como seno, coseno, exponencial, etc.
\end{itemize}

La Biblioteca estándar de C++ está disponible en cualquier compilador que cumpla con el estándar de C++. Para usarla, se debe incluir el archivo de cabecera correspondiente.
\end{frame}

\begin{frame}{Bibliotecas de la librería estandar}

\begin{itemize}
\item \texttt{<iostream>}: para entrada/salida de consola
\item \texttt{<vector>}: para el uso de vectores dinámicos
\item \texttt{<string>}: para el uso de cadenas de texto
\item \texttt{<algorithm>}: para el uso de algoritmos de ordenación, búsqueda, etc.
\item \texttt{<unordered\_map>} y \texttt{<map>}: para el uso de mapas y diccionarios
\item \texttt{<set>}: para el uso de conjuntos
\item \texttt{<cmath>}: para el uso de funciones matemáticas como \texttt{sqrt()}, \texttt{cos()}, \texttt{sin()}, etc.
\item \texttt{<chrono>}: para el uso de medidas de tiempo
\item \texttt{<thread>}: para el uso de hilos de ejecución
\item etc
\end{itemize}

Más en \url{https://en.cppreference.com/w/cpp/standard_library}

\end{frame}

\begin{frame}[fragile]
  \frametitle{Ejemplo de string y métodos}
  \begin{columns}
    \begin{column}{0.3\textwidth}
      \begin{itemize}
        \item \texttt{size()}
        \item \texttt{length()}
        \item \texttt{empty()}
        \item \texttt{clear()}
        \item \texttt{substr()}
        \item \texttt{find()}
        \item \texttt{replace()}
      \end{itemize}
    \end{column}
    \begin{column}{0.7\textwidth}
      \begin{minted}[label=string\_example.cpp]{cpp}
#include <iostream>
#include <string>
using namespace std;

int main() {
  string str = "Hello, world!";
  cout<<"\n str = "<<str;
  cout<<"\n size = "<<str.size();
  cout<<"\n substring = "<<str.substr(0, 5);
  int pos = str.find("mundo"); //pos = 5
  if(str.empty()) 
  cout << "La cadena esta vacia";
  else cout<<"La cadena tiene"<<str.length()
  <<" caracteres";

  return 0;
}
      \end{minted}
    \end{column}
  \end{columns}
\end{frame}

\begin{frame}[fragile]{Métodos de vector en C++}
\begin{columns}
\begin{column}{0.3\textwidth}
    \begin{itemize}
\item \texttt{size()}
\item \texttt{push\_back()}
\item \texttt{pop\_back()}
\item \texttt{insert()}
\item \texttt{erase()}
\item \texttt{clear()}
\item \texttt{reserve()}
\item \texttt{empty()}
    \end{itemize}
\end{column}

    \begin{column}{0.7\textwidth}
    \begin{minted}[label=vector\_example.cpp]{cpp}
#include <iostream>
#include <vector>
using namespace std;

int main() {
    vector<int> v;
    v.push_back(10);
    v.push_back(20); //{10, 20}
    v.pop_back(); //20
    v.insert(v.begin() + 1, 30); //{10, 30, 20}
    v.erase(v.begin() + 1); //{10, 20}
    v.clear();
    v.reserve(100);
    if (v.empty())
        cout << "El vector está vacío" << endl;
    return 0;
}
    \end{minted}
        \end{column}
\end{columns}

\end{frame}

\begin{frame}[fragile]{Iteradores en un vector de C++}

\begin{columns}
\begin{column}{0.5\textwidth}
Un iterador es un objeto que se utiliza para recorrer una secuencia de elementos en un contenedor, como un vector.

\begin{itemize}
    \item \texttt{begin()}: devuelve un iterador al primer elemento del vector.
    \item \texttt{end()}: devuelve un iterador al último elemento del vector.
    \item \texttt{rbegin()}: devuelve un iterador al último elemento del vector.
    \item \texttt{rend()}: devuelve un iterador al primer elemento del vector.
\end{itemize}

\end{column}

\begin{column}{0.5\textwidth}

\begin{minted}[fontsize=\small, frame=lines, label=Ejemplo]{c++}
#include <iostream>
#include <vector>
int main() {
    std::vector<int> vec = {1, 2, 3, 4, 5};
    // Recorrer el vector con un iterador
    for (auto it = vec.begin(); it != vec.end(); ++it) {
        std::cout << *it << " ";
    }
    return 0;
}
    \end{minted}
\end{column}
\end{columns}

\end{frame}


\begin{frame}[fragile]{Vectores para almacenar objetos}

\begin{minted}[label=rectangulo\_constructores.cpp]{cpp}
int main() {
    std::vector<Rectangulo> rectangulos;
    rectangulos.push_back(Rectangulo(2, 3));
    rectangulos.push_back(Rectangulo(4, 5));
    rectangulos.push_back(Rectangulo(6, 7));

    rectangulos.at(2);
    
    for (const Rectangulo& rect : rectangulos) {
        std::cout<<"Area del rectangulo: "<<rect.calcular_area()<<std::endl;
    }

    std::cout<< rectangulos.at(2).calcular_area() << std::endl;

    return 0;
}

\end{minted}
\end{frame}


\begin{frame}[fragile]{La biblioteca algorithm de C++}
\begin{columns}
\begin{column}{0.3\textwidth}
\begin{itemize}
    \item sort()
    \item find()
    \item replace()
    \item fill()
    \item max\_element()
    \item min\_element()
    \item reverse()
    \item unique()
    \item binary\_search()
    \item entre otros...
\end{itemize}
\end{column}

    \begin{column}{0.7\textwidth}
\begin{minted}[fontsize=\footnotesize, label=Ejemplo.cpp]{c++}
#include <iostream>
#include <algorithm>
#include <vector>

int main() {
    std::vector<int> v = { 3, 2, 5, 4, 1 };
    std::sort(v.begin(), v.end());
    std::cout << "Vector ordenado: ";
    for (const auto& elem : v) {
        std::cout << elem << " ";
    }
    std::cout << std::endl;
    int max = *std::max_element(v.begin(), v.end());
    std::cout << "Maximo valor del vector: " << max << std::endl;
    return 0;
}
\end{minted}
    \end{column}

\end{columns}

\end{frame}


\begin{frame}[fragile]{Lectura de archivos con stream en C++}

\begin{columns}[T]
\begin{column}{0.45\textwidth}

\begin{minted}[label=datos.txt]{c++}
123
456
789
\end{minted}
\end{column}
\begin{column}{0.55\textwidth}
\begin{minted}{c++}
#include <iostream>
#include <fstream>

int main() {
std::ifstream archivo("datos.txt");
int num;

while (archivo >> num) {
    std::cout << num << std::endl;
}

archivo.close();
return 0;
}
\end{minted}
\end{column}
\end{columns}

\end{frame}

\section{Bibliotecas no estándar}

\begin{frame}[fragile]{Boost C++ Library}

\begin{itemize}
\small
\item La biblioteca Boost es una colección de bibliotecas de software libre que extiende las capacidades de C++.
\item Boost proporciona muchas utilidades que no se encuentran en la Biblioteca Estándar de C++, cómo algoritmos geométricos
\end{itemize}

\begin{minted}{cpp}
#include <iostream>
#include <boost/geometry.hpp>
#include <boost/geometry/geometries/point_xy.hpp>
namespace bg = boost::geometry;

int main()
{
    typedef bg::model::d2::point_xy<double> point;
    point p1(1.0, 1.0), p2(4.0, 5.0);
    double distance = bg::distance(p1, p2);
    std::cout << "Distance between p1 and p2 is: " << distance << std::endl;
    return 0;
}
\end{minted}

\end{frame}

\begin{frame}{La Biblioteca CGAL}
\begin{columns}[T]
\begin{column}{0.6\textwidth}
\begin{itemize}
\item Proporciona algoritmos y estructuras de datos para problemas comunes en geometría computacional.
\item Algunas características importantes incluyen:
\begin{itemize}
\item Representación de mallas de polígonos 2D y 3D.
\item Algoritmos de triangulación, convex hull, intersecciones, etc.
\item Soporte para geometría algebraica, curvas paramétricas y superficies.
\item Herramientas para procesamiento y visualización de mallas.
\end{itemize}
\end{itemize}
\end{column}
\begin{column}{0.4\textwidth}
\begin{center}
\includegraphics[width=\textwidth]{implicit_domain.jpg}
\end{center}
\end{column}
\end{columns}
\end{frame}

\section{Tests}

\begin{frame}[fragile]{Testing con Asserts en C++}
En C++, se pueden hacer pruebas unitarias utilizando asserts. 

\begin{itemize}
\item Un assert es una macro que verifica una expresión y termina el programa si la expresión es falsa.
\item Los asserts son útiles para detectar errores lógicos en el programa, como valores inválidos de parámetros o errores de cálculo.
\end{itemize}

Para utilizar los asserts, se debe incluir la biblioteca \texttt{cassert} y luego utilizar la macro \texttt{assert} con la expresión a evaluar:

\begin{minted}{c++}
#include <cassert>

int dividir(int a, int b) {
    assert(b != 0);
    return a / b;
}
\end{minted}

Si la expresión \texttt{b != 0} es falsa, el programa terminará en ese punto y mostrará un mensaje de error.

\end{frame}

\section{Compilación}

\begin{frame}[fragile]{Compilación}
    \begin{minted}[label=makefile]{cpp}
all: maximo complejo rectangulo assert_ex iteradores rectangulo_ex1 
maximo:
	g++ maximo.cpp -o maximo
complejo:
	g++ complejo.cpp -o complejo
rectangulo:
	g++ rectangulo.cpp rectangulo.hpp -o rectangulo
rectangulo_ex1:
	g++ rectangulo_ex1.cpp -o rectangulo_ex1
assert_ex:
	g++ assert_ex.cpp -o assert_ex
iteradores:
	g++ iteradores.cpp -o iteradores
clean:
	rm -f maximo complejo rectangulo assert_ex1
    \end{minted}
\end{frame}

\begin{frame}[fragile]{CMake}
    CMake y Gtest serán explicado en la próxima auxiliar

\begin{minted}{cpp}
cmake_minimum_required(VERSION 3.5)

project(Auxiliar)

add_executable(maximo maximo.cpp)
add_executable(complejo complejo.cpp)
add_executable(rectangulo rectangulo.cpp rectangulo.hpp)
add_executable(rectangulo_ex1 rectangulo_ex1.cpp)
add_executable(assert_ex assert_ex.cpp)
add_executable(iteradores iteradores.cpp)

\end{minted}
\end{frame}

\section{Bibliografía}

\begin{frame}{Bibliography}

Stroustrup, B. (2018). A Tour of C++, Second Edition.


\begin{center}
      \includegraphics[height=8cm]{Tour3English-large.jpg}

\end{center}
    
\end{frame}

\end{document}
